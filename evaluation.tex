\section{Evaluation}
\label{sec:evaluation}

In this section we evaluate Jorvik in three different ways. 
In the first evaluation, we apply Jorvik to generate a Papyrus editor for the non-trivial Archimate UML profile~\cite{iacob2009archimate,haren2012archimate}. 
We use the Adocus Archimate for Papyrus\footnote{\url{https://github.com/Adocus/ArchiMate-for-Papyrus}} (an open-source tool that includes a profile for Archimate and the appropriate editors for Papyrus) for reference. 
We compare the proportion of the tool that Jorvik is able to generate automatically, check the number of polishing transformations that the user needs to write to complete the missing parts and finally, identify the aspects of the editor that our approach is not able to generate.
As a result we can measure the \textit{efficiency} of Jorvik in generating profiles/editors against an existing relatively large profile/editor. 

In the second evaluation, we assess the \textit{completeness} of Jorvik by applying it to a number of metamodels collected as part of the work presented in~\cite{williams2013metamodels}. 
This way, Jorvik is tested to check if it can successfully generate profiles and editors for a wide variety of scenarios.

In the third evaluation, we conduct a \textit{user experiment} in which we asked software engineers to build Papyrus editors for two UML Profiles. 
We first ask the engineers to create the profiles and editors manually, and then ask them to create the same profiles and editors using Jorvik. 
We measure the time, report problems encountered during the experiment for both approaches and we compare the results.

\subsection{Efficiency}
\label{sec:efficiencyEvaluation}
The Archimate for Papyrus tool offers five kinds of diagrams (i.e., Application, Business, Implementation and Migration, Motivation and Technology diagrams). 
Each of the diagrams uses different stereotypes from the Archimate profile. 
In this scenario, we create five Ecore metamodels and annotate the elements that need to appear as nodes/edges in the diagrams. 
We then generate the editors for all the five Archimate diagrams.
At this point, five fully functional editors are generated that can be used to create each of the five types of diagrams that the Archimate for Papyrus tool also supports. 

However, our generated editors do not offer some special features that the Archimate for Papyrus tool offers. 
For example, Archimate for Papyrus offers a third drawer in the palette for some diagrams that is called ``Common'' and includes two tools (named ``Grouping'' and ``Comment''). 
Another feature that is not supported by our default transformations is the fact that in Archimate for Papyrus, users are able to have the elements represented either by their shapes or by a coloured rectangle depending on the CSS class applied to them. 
Finally, Archimate for Papyrus also organises the creation of the \textit{Junction} (which is a node that acts as a junction for edges) node in the relations' drawer in the palette.
In order to be able to implement such missing features, we need to write the extra polishing transformations. 
We do not go into details on the content of the polishing transformations for this specific example\footnote{The generated Plug-ins for Archimate and the polishing transformations are available from \url{https://github.com/wrwei/Jorvik}}.

In our previous work \cite{zolotas2018towards}, we compared our approach with Archimate for Papyrus. 
However, as we mentioned in Section~\ref{sec:background}, Papyrus changed its underlying metamodels and the mechanism for creating UML specific editors. 
To ensure that our results are still valid for Papyrus 3.0+, we re-generated all the Archimate editors using Jorvik. 
We add the lines of code needed for Jorvik to our findings in the previous work.

Table~\ref{tab:evaluation} summarises the efficiency of Jorvik, both for Jorvik pre-Papyrus 3.0 version and for Jorvik post-Papyrus 3.0 version\footnote{Cells in gray are artefacts not needed for implementation. E.g. Creation Command and Architecture Model are concepts in Papyrus version 3.0+, and therefore are not applicable to Jorvik pre-Papyrus 3.0 version and Archimate for Papyrus}. The numbers are shown in the format of \textit{Lines of Code (Number of Model Elements)}, as we count both the lines of code and (equivalent) number of model elements needed to be manually created. For artefacts which are not models (e.g., the CSS file) we only provide the lines of code metric as well for artefact created by polishing transformations in Jorvik, as these were generated by the polishing transformation scripts.

For Jorvik pre-Papyrus 3.0 (columns under \textit{Jorvik (pre-Papyrus 3.0)}), we need to manually create five annotated Ecore metamodels, which involves writing 436 lines of code (668 model elements). 
For polishing transformations, we need to write 11 lines of code in the transfornation scripr for Element Types Configuration, 50 lines for Palette Configuration, 195 lines for CSS and 10 lines for Types Configuration.

For Jorvik post-Papyrus 3.0 (columns under \textit{Jorvik (post-Papyrus 3.0)}), we need the same Ecore metamodels, thus the numbers do not change. 
For polish transformations, we need to write 50 lines for the Palette Configuration and 195 lines for CSS. 

As it can be observed from the numbers, our approach requires about 90\% less handwritten code to produce the basic diagrams and about 85\% less code to 
produce the polished editor that matches the original Archimate for Papyrus editor. 
%The produced by Jorvik editors match those included in the original Archimate for Papyrus tool.
Our approach creates the five diagram editors which offer the same functionality and features as the original Archimate for Papyrus tool but also atop that the ETL transformation and the OCL constraints.

\captionsetup{justification=centering}
\begin{landscape}
	\topskip0pt
%	\vspace*{\fill}
\begin{table}[htb!]
	\centering
	\setlength{\tabcolsep}{3.5pt} 
	\begin{tabular}{M{1.5cm}|M{1.8cm}|M{1.8cm}|M{1.8cm}|M{1.8cm}|M{1.8cm}|M{1.8cm}|M{3cm}|}
		\cline{2-8}
		& \multicolumn{3}{M{5.4cm}|}{\textbf{Jorvik (pre-Papyrus 3.0)} LoC~(Number of Model Elements)} & \multicolumn{3}{M{5.1cm}|}{\textbf{Jorvik (post-Papyrus 3.0)}  LoC~(Number of Model Elements)} & \textbf{Archimate for Papyrus (Pre Papyrus 3.0)}\\ \hline
		\multicolumn{1}{|M{2cm}|}{\textbf{File}} & \textbf{Hand-written} & \textbf{Hand-written (Polishing)} & \textbf{Total} & \textbf{Hand-written} & \textbf{Hand-written (Polishing)} & \textbf{Total} & \textbf{Total Hand-written}\\ \hline
		\multicolumn{1}{|M{2cm}|}{\textbf{ECore}} & 436 (668) & 0 & 436 (668) & 436 (668) & 0 & 436 (668) & 0 \\ \hline
		\multicolumn{1}{|M{2cm}|}{\textbf{Profile}} & 0 & 0 & 0 & 0 & 0 & 0 & 1867 (1089) \\ \hline
		\multicolumn{1}{|M{2cm}|}{\textbf{Element Types Configuration}} & 0 & 11 & 11 & 0 & 0 & 0 & 237 (61) \\ \hline
		\multicolumn{1}{|M{2cm}|}{\textbf{Palette Configuration}} & 0 & 50 & 50 & 0 & 50  & 50  & 1305 (323) \\ \hline
		\multicolumn{1}{|M{2cm}|}{\textbf{CSS}} & 0 & 195 & 195 & 0 & 195 & 195 & 537 \\ \hline
		\multicolumn{1}{|M{2cm}|}{\textbf{Creation Command}} & \cellcolor{Gray} & \cellcolor{Gray} & \cellcolor{Gray} & 0 & 0 & 0 & \cellcolor{Gray}  \\ \hline
		\multicolumn{1}{|M{2cm}|}{\textbf{Architecture Model}} & \cellcolor{Gray} & \cellcolor{Gray}& \cellcolor{Gray} & 0 & 0 & 0 & \cellcolor{Gray}  \\ \hline
		\multicolumn{1}{|M{2cm}|}{\textbf{Types Configuration}} & 0 & 10 & 10 & \cellcolor{Gray} & \cellcolor{Gray}& \cellcolor{Gray} & 788 (327) \\ \hline
		\multicolumn{1}{|M{2cm}|}{\textbf{Diagram Configuration}} & 0 & 0 & 0 & \cellcolor{Gray} & \cellcolor{Gray}& \cellcolor{Gray} & 58 (28) \\ \hline
		%\multicolumn{1}{|M{2cm}|}{\textbf{plugin.xml}} & 0 & & & 82 \\ \hline
		%\multicolumn{1}{|M{2cm}|}{\textbf{MANIFEST.MF}} & 0 & & & \\ \hline
		\multicolumn{1}{|M{2cm}|}{\textbf{Total}}  & \textbf{436 (668)} &\textbf{266} & \textbf{702 (668)} & \textbf{436 (668)} & \textbf{245} & \textbf{681 (668)} & \textbf{4792 (1828)} \\ \hline
		\cline{1-8}
	\end{tabular}
	\caption{Lines of manually written code of each file for creating a Papyrus UML profile and editor for ArchiMate.}
	\label{tab:evaluation}
\end{table}
\end{landscape}

\subsubsection{Threats to Validity}
There were a few minor features of the original Archimate for Papyrus tool that our approach could not support. 
Most of them are related to custom menu entries and wizards. For those to be created developer needs to extend the ``plugin.xml'' file. 
In addition, the line decoration shapes of stereotypes that extend the aggregation base element (i.e., diamond) can only be applied dynamically by running Java code that will update the property each time the stereotype is applied. 
Our default and polishing transformations are not able to generate those features automatically; these should be implemented manually. 
For that reason, we \textit{excluded} these lines of code needed by Archimate for Papyrus to implement these features from the data provided in Table~\ref{tab:evaluation} for a fair comparison. 

\subsection{Completeness}
\label{sec:completenessEvaluation}
In addition to the generation of the Archimate profile/editors, we tested Jorvik with nine more Ecore metamodels from different domains. 
The names of the metamodels (including Archimate) and their size (in terms of types) are given in Table~\ref{tab:metamodels}. 
Next to the size, in parenthesis, the number of types that should be transformed so they can be instantiated as nodes/edges is also provided.

\begin{table}[ht!]
	\centering
	\setlength{\tabcolsep}{3.5pt} 
	\begin{tabular}{|c|M{3cm}|c|M{3cm}|}
		\cline{1-4}
		\textbf{Name}  & \textbf{\#Types (\#Nodes/\#Edges)} & \textbf{Name}  & \textbf{\#Types (\#Nodes/\#Edges)}\\ \hline
		\textbf{Professor} & 5 (4/5)  & \textbf{Ant Scripts} & 11 (6/4) \\ \hline
		\textbf{Zoo} & 8 (6/4) & \textbf{Cobol} & 13 (12/14) \\ \hline
		\textbf{Usecase} & 9 (4/4) & \textbf{Wordpress} & 20 (19/18)  \\ \hline
		\textbf{Conference} & 9 (7/6) & \textbf{BibTeX} & 21 (16/2) \\ \hline
		\textbf{Bugzilla} & 9 (7/6) & \textbf{Archimate} & 57 (44/11) \\ \hline
		\cline{1-4}
	\end{tabular}
	\caption{The names and sizes of the ten metamodels against which the approach was evaluated to test completeness}
	\label{tab:metamodels}
\end{table}

As illustrated in Table~\ref{tab:metamodels}, the metamodels varied in size, from small profiles (having 5 stereotypes) to large profiles (up to 57 
stereotypes). 
The approach was able to produce the profiles and the editors for \textit{all} the metamodels, demonstrating that it can be used to generate the 
desired artifacts for a wide spectrum of domains. 
The time needed for the generation varied from miliseconds up to a few seconds. 
In the future, we plan to assess further the scalability of our approach using larger metamodels.

\subsection{User Experiment}
We have argued that Jorvik provides significant gains in productivity when building custom UML Profile editors for Papyrus.
We design a user experiment to substantiate such claim, and ascertain how significant the productivity gain can be.
As discussed in Section~\ref{sec:implementation}, there are eight major steps to be taken in order to create a UML profile as well as its supporting editor. 
In this experiment, we compare the time needed to develop an editor using the Papyrus infrastructure (referred to as \textit{Papyrus approach from now on}) with the time need to develop the same editor using Jorvik.
For the Papyrus approach we design eight tasks, each with its own deadline (see Table~\ref{tab:manual}) for the participants to complete towards manually creating a UML profile and a working UML editor for the profile.
For Jorvik, we designed one task for the participants to complete to automatically generate a UML profile and a working UML editor for that profile.
We ask two participants to take part in the experiment and work on two profiles we choose. 
We record the time taken for the participants to complete the experiment using both approaches and we compare the times.

\subsubsection{Papyrus Approach Experiment Set-Up}
For the purpose of this experiment, we have chosen a participant with relatively more experience in modelling, and a candidate with less experience in modelling. 
Both participants have an Eclipse IDE installed on their computers, with Eclipse Epsilon 1.6 Interim version\footnote{\url{https://www.eclipse.org/epsilon/}} and Eclipse Papyrus 4.0.0\footnote{\url{https://www.eclipse.org/papyrus/download.html}} installed.
We asked the participants to perform the eight tasks involved in the Papyrus apporach firstly on one profile (the Website profile\footnote{\url{https://github.com/wrwei/Jorvik/tree/master/org.papyrus.website}}) and then repeat the experiment for a second profile (the Fault Tree profile\footnote{\url{https://github.com/wrwei/Jorvik/tree/master/org.papyrus.faulttree}}) following again the Papyrus apporach. 
The SVG shapes and icons for both the case were provided to the participants. 
Before the experiment is conducted, a pre-experiment questionnaire is handed to the participants, to assess their expertise in UML, UML profiles and Papyrus. \thanos{Refer}
In addition, a 20-30 minutes introduction to UML profiles and Papyrus is given to them while an example of a custom UML profile Papyrus editor is being presented to them. 

For each of the eight steps, there is a set deadline; the participants are asked to try to complete the step within the deadline.
The tasks and the deadlines are derived from our own experience in developing a working UML profile specific Papyrus editor. 
Initially we spent 3 months on creating an example editor, due to the lack of documentation, and the lack of tool support when referencing model elements among the models required for the editor.
After we found out how to create an editor, we recorded the amount of time required for us to perform the eight steps to derive the deadlines. 
We then normalise the deadlines through a pilot study with a volunteer from our research group (we also make adjustments to our experiment set-up in the pilot study based on what we learnt from it). 

In each step, the participants are asked to complete a minimal task first (e.g. for UML profile, create a \textit{Stereotype} that is displayed as a node and a \textit{Stereotype} that is displayed as an edge)\footnote{Detailed descriptions of the tasks can be found at \url{https://github.com/wrwei/Jorvik/tree/master/User\%20Experiment}}. 
They are asked to continue with the rest if there is still time left, while, if the deadline is missed but they were close to the correct solution, they are asked to give an estimate of how long they believe it would take them to finish the whole step.
At the beginning of each step, we provide a piece of \textit{Default} knowledge, which covers ground knowledge for the step to be completed. 
Participants are also allowed to search for any information over the Internet which may assist them in their tasks at any point during the experiment.
At a certain point for each step, we assess if the participants are able to complete the step within the time frame, and we provide a piece of \textit{Essential} knowledge, which contains key information (parts of which are not easily accessible from the Internet) for the participants to complete the step.

\begin{table}[ht!]
	\centering
	\setlength{\tabcolsep}{3.5pt} 
	\begin{tabular}{|c|c|c|c|}
		Task & Total (m) & Default (m) & Essential (m) \\ 
		1. UML Profile & 60 & 40 & 20 \\
		2. Element Types Configuration Model & 60 & 40 & 20 \\
		3. Palette Configuration Model & 30 & 20 & 10 \\
		4. Cascading Style Sheet & 30 & 20 & 10 \\
		5. Creation Command & 30 & 20 & 10 \\
		6. Architecture Model & 40 & 30 & 10 \\
		7. Plug-in Configuration & 20 & 12 & 8 \\
		8. OCL Constraints & 60 & 40 & 20 \\
	\end{tabular}
	\caption{Tasks and times for the Papyrus approach.}
	\label{tab:manual}
\end{table}


The overview of the tasks is provided in Table~\ref{tab:manual}. The detailed task descriptions are as follows:
\begin{enumerate}
	\item UML Profile - An image of a UML profile is provided to the participants, they are required to create the profile within 60 minutes.
	\textit{Essential} information is provided at minute 40.
	\item Element Types Configuration Model - Participants are asked to create an Element Types Configuration model for the editor within 60 minutes. 
	\textit{Essential} information is provided at minute 40.
	\item Palette Configuration Model - Participants are asked to create a Palette Configuration model for the editor within 30 minutes.
	\textit{Essential} information is provided at minute 20.
	\item Custom Style - Participants are asked to create a CSS file to customise the styles of the editor within 30 minutes.
	\textit{Essential} information is provided at minute 20.
	\item Creation Command - Participants are asked to create a Java class to initialise the Papyrus diagram within 30 minutes.
	\textit{Essential} information is provided at minute 20.
	\item Architecture Model - Participants are asked to create an Architecture model for the editor within 40 minutes.
	\textit{Essential} information is provided at minute 30.
	\item Plug-in Configuration - Participants are asked to configure their plug-ins in order to make use of all the models/artefacts to form a working editor, within 20 minutes.
	\textit{Essential} information is provided at minute 12.
	\item OCL constraints (optional) - In this optional task, participants are asked to create OCL constraints mentioned in Section~\ref{sec:constraints} for all connector \textit{Stereotype}s, within 60 minutes. 
	We do not expect this task to be taken by participants, as it typically required experienced OCL experts 2 weeks to complete the constraint templates described in Section~\ref{sec:constraints}.
	For the record, no participants agreed to take this optional task.
\end{enumerate}

For each step, when participants express that they have completed the step, we stop the timer and assess the solutions. 
If the solutions are not correct, we tell the participants that they need more work and resume the timer.
Since each step depends on the previous being completed, at the beginning of each step, the participants are given a solution that contains complete and correct assets for all the previous tasks.

After participants have completed the manual process, a post-experiment questionnaire is handed to them, in which they evaluate the difficulties of each task and if enough time was provided for each task.

\begin{table}[ht!]
	\centering
	\setlength{\tabcolsep}{3.5pt} 
	\begin{tabular}{|c|c|c|c|}
		Task & Total (m) & Default (m) & Essential (m) \\ 
		1. Ecore Metamodel and Generation & 50 & 30 & 20 \\
	\end{tabular}
	\caption{Tasks and times for the automatic process.}
	\label{tab:automatic}
\end{table}

\subsubsection{Jorvik Experiment Set-up}
The experiment then proceeds to the use of Jorvik, where the participants need to complete one task (see Table~\ref{tab:automatic}):
\begin{enumerate}
	\item Annotated Ecore Metamodel and Generation - Participants are provided with the same image of the profiles, they are asked to create an annotated Ecore metamodel for the profile, and generate the UML profile and the editor, within 50 minutes.
	The \textit{Default} information is provided at the beginning. 
	30 minutes in the task, we assess the participants' status and provide the \textit{Essential} information.
\end{enumerate}



\begin{landscape}


\begin{table}
\begin{center}
	\centering
    \begin{tabular}{ l | l | M{2.8cm} | c c c c c | c |}
    \cline{2-9}
    \multirow{2}{*}{} & \multirow{2}{*}{Task} & Time Given & \multicolumn{2}{c}{Time Taken} & \multicolumn{2}{|c|}{Correctness} &  \multirow{2}{*}{Participant} &  \multirow{2}{*}{Remarks} \\
    & & Default (Essential) & Web & FTA &  \multicolumn{1}{|c}{Web} &  \multicolumn{1}{c|}{FTA} & & \\ \cline{2-9}
    \multirow{15}{*}{\rotatebox[origin=c]{90}{Papyrus}} & \multirow{ 2}{*}{1. UML Profile} & \multirow{ 2}{*}{40m (20m)} & 25m 30s & 15m & \multicolumn{1}{|c}{\CIRCLE} & \multicolumn{1}{c|}{\CIRCLE} & \#1 & N/A \\
   	& & & 40m & 24m & \multicolumn{1}{|c}{\LEFTcircle} & \multicolumn{1}{c|}{$\CIRCLE$} & \#2 & N/A \\ \cline{2-9}
   	
   	& \multirow{ 2}{*}{2. Element Types Configuration} & \multirow{ 2}{*}{40m (20m)} & 56* & 38m & \multicolumn{1}{|c}{$\CIRCLE$} & \multicolumn{1}{c|}{$\CIRCLE$} & \#1 & \textcircled{1}\\
   	& & & 60m* & 58m* & \multicolumn{1}{|c}{$\Circle$} & \multicolumn{1}{c|}{$\Circle$} & \#2 & \textcircled{2} \\ \cline{2-9}
   	
   	& \multirow{ 2}{*}{3. Palette Configuration} & \multirow{ 2}{*}{20m (10m)} & 26m30s* & 25m & \multicolumn{1}{|c}{$\CIRCLE$} & \multicolumn{1}{c|}{$\CIRCLE$} & \#1 & \textcircled{3} \\
   	& & & 30m* & 30m* & \multicolumn{1}{|c}{$\Circle$} & \multicolumn{1}{c|}{$\LEFTcircle$} & \#2 & \textcircled{4} \\ \cline{2-9}
   	
   	& \multirow{ 2}{*}{4. Custom Style} & \multirow{ 2}{*}{20m (10m)} & 15m30s & 16m & \multicolumn{1}{|c}{$\LEFTcircle$} & \multicolumn{1}{c|}{$\CIRCLE$} & \#1 & N/A \\
   	& & & 30m* & 19m & \multicolumn{1}{|c}{$\LEFTcircle$} & \multicolumn{1}{c|}{$\LEFTcircle$} & \#2 & N/A \\ \cline{2-9}
   	
   	& \multirow{ 2}{*}{5. Creation Command} & \multirow{ 2}{*}{20m (10m)} & 25m* & 16m & \multicolumn{1}{|c}{$\Circle$} & \multicolumn{1}{c|}{$\CIRCLE$} & \#1 & \textcircled{5} \\
   	& & & 30m* & 25m & \multicolumn{1}{|c}{$\Circle$} & \multicolumn{1}{c|}{$\CIRCLE$} & \#2 & N/A \\ \cline{2-9}
   	
   	& \multirow{ 2}{*}{6. Architecture Model} & \multirow{ 2}{*}{30m (10m)} & 40m* & 25m & \multicolumn{1}{|c}{$\Circle$} & \multicolumn{1}{c|}{$\CIRCLE$} & \#1 & \textcircled{6} \\
   	& & & 40m* & 40m* & \multicolumn{1}{|c}{$\Circle$} & \multicolumn{1}{c|}{$\LEFTcircle$} & \#2 & \textcircled{7} \\ \cline{2-9}
   	
   	& \multirow{ 2}{*}{7. Plug-in Configuration} & \multirow{ 2}{*}{12m (8m)} & 20m* & 5m & \multicolumn{1}{|c}{$\Circle$} & \multicolumn{1}{c|}{$\CIRCLE$} & \#1 & \textcircled{8} \\
   	& & & 19m & 10m & \multicolumn{1}{|c}{$\CIRCLE$} & \multicolumn{1}{c|}{$\CIRCLE$}& \#2 & N/A \\ \cline{2-9}
   	
   	& \multirow{2}{*}{Total Time} & \multirow{2}{*}{182m (88m)} & 208m30s & 140m & \multicolumn{1}{|c}{} &\multicolumn{1}{c}{}& \multicolumn{1}{c}{}&\multicolumn{1}{c}{} \\ 
   	& & &  249m & 206m & \multicolumn{1}{|c}{} &\multicolumn{1}{c}{}& \multicolumn{1}{c}{}&\multicolumn{1}{c}{} \\ \cline{2-9}
   	
   	\multirow{2}{*}{Jorvik} & \multirow{ 2}{*}{EMF + Annotations} & \multirow{ 2}{*}{30m (20m)} & 32m 30s* & 14m 51s & \multicolumn{1}{|c}{\CIRCLE} & \multicolumn{1}{c|}{\CIRCLE} & \#1 & N/A \\
   	& & & 36m 55s* & 37m 27s* & \multicolumn{1}{|c}{$\CIRCLE$} & \multicolumn{1}{c|}{$\CIRCLE$} & \#2 & N/A \\ \cline{2-9}
	\end{tabular}
    \newline 
    \\ {\bf Legend} \\
   
\begin{tabular}{|c|c|c|c|}
\hline
$\CIRCLE$ : Correct & $\LEFTcircle$ : Partially correct & $\Circle$: Incorrect & * : Essential Information Given\\ \hline
\end{tabular} 
\caption{Evaluation of Requirement Specification Languages as shown in \cite{tse1991rsl}}
\label{tab:experiment}
\end{center}
\end{table}
\end{landscape}


\subsubsection{Results}
Table~\ref{tab:experiment} shows the times obtained from the user experiment.
Participant \#1 is a female PhD student who has a high level of expertise in modelling, and has used Papyrus before.
Participant \#2 is a male PhD student who has an intermediate level of expertise in modelling, and has used Papyrus on a limited number of occasions. 
Both of the participants have no experience in creating a distributable editor for UML profiles.

In the table, the \textit{Task} column specifies the name of the steps. 
The \textit{Time Given} column specifies the time we give the participants for each step, the format \textit{X (Y)} means that we provide the \textit{Default} knowledge at the beginning and we time for X minutes, then provide the \textit{Essential} and we time for Y minutes, then we stop the participants.
The \textit{Participant Time} column records the time taken for the participants to complete the Website profile (\textit{Web}) and the Fault Tree profile (\textit{FTA}). 
The \textit{Correctness} column records if the participants are able to provide correct solutions. 
The participants are distinguished using the \textit{Participant} column.
We record any comments/remarks made by the participants in the \textit{Remarks} column.

Below is an example of how the table should be read (the summary of the experiment for Participant \#1 for the Papyrus approach for the Website profile):
\begin{enumerate}
	\item She was able to finish the UML profile creation in 25 minutes without the essential information. 
	\item She finished the Element Types Configuration in 56 minutes with the help of the essential information. 
	
%	\textit{Remark \textcircled{1}:} She claimed that she found a solution online.\footnote{Which is the forum thread where the authors obtained the correct way of creatin Element Types Configurations: \url{https://www.eclipse.org/forums/index.php/t/1096471/}}.
	\item She could not figure out how to complete the Palette Configuration model, we provided her with the essential information, which helped her finish the model in 26 minutes (for the minimal task). 
	
%	\textit{Remark \textcircled{3}:} She claims that she would need 20 more minutes to finish the model.
	\item She finished a partial solution for the CSS in 15 minutes.
	\item She could not figure out how to create a creation command, therefore essential information is provided, and she finished the step in 25 minutes in total.
	
%	\textit{Remark \textcircled{5}:} She copied the actual solution for the essential information given.
	\item She could not figure out how to create an Architecture model, even with the essential information provided and missed the deadline.
	\item She was not able to configure the editor P lug-in to successfully run the editor, even with the essential information. 
	
%	\textit{Remark \textcircled{8}: } She claims that she would need 30+ minutes more to finish the configuration.
\end{enumerate}

%In total Participant \#1 spent 208 minutes 30 seconds on the manual process for the Website profile.
%
%Below is the summary for Participant \#1 using the Papyrus approach for the Fault Tree profile:
%
%\begin{itemize}
%	\item She was able to finish the UML profile creation in 15 minutes without the essential information. 
%	\item She finished the Element Types Configuration in 38 minutes without the help of the essential information.
%	\item She finished the Palette Configuration model in 25 minutes without the help of the essential information.
%	\item She finished a complete solution for the CSS in 16 minutes.
%	\item She finished the creation command Java class in 16 minutes.
%	\item She finished the architecture model in 25 minutes.
%	\item She was able to configure the editor plug-in to successfully run the editor in 10 minutes.
%\end{itemize}
%\thanos{I think that we need to make a remark here that in many tasks she copy-paste part of the solution from the Website profile and adapted it}
%In total Participant \#1 spent 140 minutes using the Papyrus approach for the Fault Tree profile.\thanos{Again participant \#1, right?}
%
%Below is the summary for Participant \#2 for the Papyrus approach for the Website profile:
%\begin{enumerate}
%	\item He was able to finish the UML profile creation in 40 minutes without the essential information. 
%	\item He could not finish the Element Types Configuration model, even with the help of the essential knowledge. 
%	\item He could not figure out how to complete the Palette Configuration model, even with the essential knowledge. 
%	\item He finished a partial solution for the CSS in 30 minutes.
%	\item He could not figure out how to create a creation command, therefore essential information is provided, and he finished the step in 30 minutes in total.
%	\item He could not figure out how to create an Architecture model, even with the essential information provided and he still missed the deadline. 
%	
%	\textit{Remark \textcircled{6}:} He estimated that he would need 60+ minutes more to finish the model.
%	\item He was able to configure the editor plug-in to successfully run the editor in 19 minutes.
%\end{enumerate}
%In total Participant \#2 spent 249 minutes seconds using the Papyrus approach for the Website profile.
%
%Below is the summary for Participant \#2 for the Papyrus approach for the Fault Tree Profile:
%
%\begin{enumerate}
%	\item He was able to finish the UML profile creation in 24 minutes without the essential information. 
%	\item He finished the minimal task for the Element Types Configuration in 58 minutes with the help of the essential information. 
%	
%	\textit{Remark: \textcircled{2}} He would need 40+ minutes to complete the whole model.
%	\item He finished the Palette configuration model in 22 minutes with the help of essential however the solution was not correct.
%	
%	\textit{Remark \textcircled{4}:} \thanos{Text is missing for this remark.}.
%	\item He finished a complete solution for the CSS in 19 minutes.
%	\item He finished the creation command Java class in 25 minutes.
%	\item He finished the architecture model in 32 minutes with the help of the essential knowledge. 
%	
%	\textit{Remarks: \textcircled{7}: The solution was incorrect.}
%	\item She was able to configure the editor plug-in to successfully run the editor in 10 minutes.
%\end{enumerate}
%In total Participant \#2 spent 206 minutes using the Papyrus approach for the Fault Tree profile.


We also receive/observe some interesting remarks during the experiment.
Below is a list of the description of the remarks in the table:
\begin{itemize}
	\item Remark \textcircled{1}: In the Website experiment, in Step 2, Participant 1 claimed that she found a solution online\footnote{Which is the forum thread where the authors obtained the correct way of creatin Element Types Configurations: \url{https://www.eclipse.org/forums/index.php/t/1096471/}.} that made the task significantly easier. 
	She also claims that without the solution there is no way she could have finished the task, even with the \textit{Essential} knowledge.
	\item Remark: \textcircled{2} In both the Website and the Fault Tree experiment, in Step 2, Participant 2 claimed that he could never complete the step, without the essential information. 
	He also claims that the Element Types Configuration is rather confusing.
	In the Fault Tree experiment, he claimed he would need 40+ minutes to complete the whole model.
	\item Remark \textcircled{3}: In the Website experiment, in Step 3, Participant 1 finished the minimal task with the help of the essential information. She	claimed that she would need 20 more minutes to finish the model.
	\item Remark \textcircled{4}: In the Website experiment, in Step 3, Participant 2 missed the deadline even with the essential information.
	He claimed that he would need 30 more minutes to finish the step.
	\item Remark \textcircled{5}: In the Website experiment (and presumably in the Fault Tree experiment), in step 5, participant claims that she copied the actual solution for the essential information given.
	\item Remark \textcircled{6}: In the Website experiment, in step 6, participant 1 missed the deadline even with the essential information. 
	She claims that the tool support for the Architecture model by Papyrus is not well implemented (it does not support the reference to model elements in other models).
	\item Remark \textcircled{7}: Participant 2 in both experiment claimed that he finished the step before the deadline (both with the help of the essential information), but he could not get the solutions correct. 
	This is typically due to the fact that there are somewhat confusing model elements in the Architecture metamodel by Papyrus.
	\item Remark \textcircled{8}: In the Website experiment, in step 7, Participant 1 could not configure the Plug-in to a working order, she claimed that she would need more than 20 minutes to inspect other models to find out what went wrong.
\end{itemize}

Using Jorvik, participant \#1 was able to generate the correct Papyrus editor for the Website profile in 32 minutes. She was also able to create the correct editor for the Fault Tree profile in 15 minutes.
Participant \#2 needed 37 minutes and 38 minutes for the creation of a correct Papyrus editor for the Website and the Fault Tree profiles, respectively, using Jorvik.


\subsubsection{Analysis} 
We begin our analysis with the responses from the participants for the pre-experiment questionnaires.
Both participants in the questionnaire express that they have intermediate knowledge of UML. 
Prior to the experiment, both have not created a UML profile in the past.
Both have not experience with Papyrus, and therefore have not created UML profiles using Papyrus.
We expect this would be the case as Papyrus UML profile is not the most used function of Papyrus.

We then move on to the analysis of the responses to the post-experiment questionnaires. \will{Maybe Horacio can provide some insights?}


Comparing the Papyrus approach with Jorvik, we can conclude that using Jorvik users are able to increase the productivity by at least 10 times. 
We draw this conclusion based on the data presented in Table~\ref{tab:experiment} (especially when participants claim that they would need additional time to finish the complete solution for some steps), and also based on the fact that both participants chose not to complete the optional Step 8 (the OCL constraints), which may take significant time, even for experience OCL programmers.

We are also able to draw the conclusion that it is rather difficult to derive the models/artefacts needed for a working UML profile-specific editor. 
This is based on the experiment results that both participants got the majority of their models/artefacts wrong for the Website profile, which was the first profile and editor they worked on.
Although we provided the \textit{Essential} information, (which to the best of our knowledge, is not available in Papyrus documentations) the participants still could not get the models/artefacts right because of the inter-related nature of the models.
For example, both participants find it difficult to comprehend the purpose of the Element Types Configuration, they actually find that it is the most challenging part of the experiment.
The candidates also find it difficult to link creation tools in the Palette Configuration model to elements in the Element Types Configuration. 
They also find it hard to understand the rationale behind referencing to Element Types Configuration.
And finally, the participants both claim that it is rather difficult to create the Architecture model, for there are concepts defined in the Architecture model which meanings are not orthogonal to their experience.
This is the direct evidence that creating a distributable UML profile editor in Papyrus is a labour-intensive and error-prone process.
In addition, due to the inter-connected nature among the models/artefacts, the participants find it difficult to debug their solutions, for there are many places where things may go wrong.
In contrast, Jorvik provides detailed feedback based on the validation rules applied to the annotated Ecore metamodel, which makes it easier for the participants to debug.
In the Jorvik experiment, we noticed that the participants typically made use of the feedback provided by Jorvik to debug their annotated Ecore metamodels.

When participants worked on the Fault Tree profile, they were able to refer to their solutions to the Website profile. 
Therefore we observe that the correctness of the models/artefacts for the Fault Tree profile improved significantly comparing to the Website profile. This matches our experience with using the Papyrus infrastructure for the development of distributable editors for UML profiles, where we had to reverse engineer other editors available online to try to understand how to proceed.
Although they may have adapted their Website soutions to their Fault Tree solution, the recorded time shows that Jorvik still dominates the manual process, especially taking the OCL constraints into consideration.

It is worth mentioning that Participant \#1 is an expert user of Ecore, where Participant \#2 only used Ecore occasionally. 
We observe the advantage of being familiar with Ecore based on Participant \#1's time taken for the experiments (especially after she got familiar with the annotation rules for Jorvik).
However, the level of expertise in Ecore does not affect the results of our evaluation.


\subsubsection{Threats to Validity}
For the experiment we could not find experts that specialise in creating UML profiles and their supporting editors for Papyrus. 
This is typically due to the fact that this feature of Papyrus is not widely promoted and used.
However, we have manually created the Website profile and its editor and recorded the time it took us to finish the whole solution. 

\textbf{[TODO: Other threats: 1) People were asked to execute tasks in a waterfall way. Normally, when we do this proceess, we go back and forth between the artefacts, make changes, test, fix, go back, make changes in antoher model etc. etc. ]}



[\textbf{TODO: General remark: 1) Pilot study is not referred anywhere. It should be referred in the experiment setup to highlight that the experiment was refined. 2) The results of the questionnaire are not presented and discussed. I also noticed tha the questionnaires appendix is missing.}]
