\section{Evaluation}
\label{sec:evaluation}

In this section we evaluate AMIGO in two different ways. 
Firstly, we apply it to generate a Papyrus editor for the non-trivial Archimate 
UML profile~\cite{iacob2009archimate,haren2012archimate}. The Adocus Archimate 
for Papyrus\footnote{\url{https://github.com/Adocus/ArchiMate-for-Papyrus}} is 
an open-source tool that includes a profile for Archimate and the appropriate 
editors for Papyrus. This way, we can compare the proportion of the tool that 
AMIGO is able to generate automatically, check the number of polishing 
transformations that the user needs to write to complete the missing parts and 
finally, identify the aspects of the editor that our approach is not able to 
generate. As a result we can to measure the \textit{efficiency} of AMIGO in generating profiles/editors against an existing relatively large 
profile/editor. 

Secondly, we assess the completeness of our approach by applying it on a number 
of other metamodels collected as part of the work presented 
in~\cite{williams2013metamodels}. This way, the approach is tested to check if 
it can successfully generate profiles and editors for a wide variety of 
scenarios.

\subsection{Efficiency}
\label{sec:efficiencyEvaluation}
The Archimate for Papyrus tool offers five kind of diagrams (i.e., Application, Business, Implementation and Migration, Motivation and Technology diagrams). Each of the diagrams uses different stereotypes from the Archimate profile. 

As shown in Figure~\ref{fig:approachOverview}, our approach starts by 
implementing the Ecore metamodel for the profile/editor we would like to 
generate. Thus, in this scenario we need to create the 5 Ecore metamodels and 
annotate those EClasses/EReferences that need to appear as nodes or edges on 
the diagrams. The profiles for each type of Archimate diagram and the editors 
can now be generated. At this point, five fully functional editors are 
generated that can be used to create each of the five types of diagrams that 
the Archimate for Papyrus tool also supports. 

Nevertheless, our generated editors do 
not offer some special features that the Archimate for Papyrus tool offers. For 
example, the latter offers a third drawer in the palette for some diagrams that 
is called ``Common'' and includes two tools (i.e., ``Grouping'' and 
``Comment''). Another feature that is not supported by our default 
transformations is the fact that in Archimate for Papyrus, users are able to 
have the elements represented either by their shapes or by a coloured rectangle 
depending on the CSS class applied to them. In order to be able to implement 
such missing features, we need to write the extra polishing transformations. 
For brevity, we will not go into details on the content of the polishing 
transformations for this specific example.

Table~\ref{tab:evaluation}, summarises the lines of code we had to write manually to generate each file needed by Papyrus to create the 5 basic diagrams (column ``AMIGO'' / ``Handwritten'') and the number of lines of the polishing transformations that were needed to produce the enhanced editor (column ``AMIGO'' / ``Handwritten (Polishing)''). The totals are given in column ``AMIGO'' / ``Total''. For easier comparison, the total lines of code the authors of the Archimate for Papyrus had to write manually are provided in column ``Archimate for Papyrus'' / ``Total Handwritten''\footnote{The produced plugins generated with AMIGO for the Archimate profile can be downloaded from \url{http://www.zolotas.net/AMIGO/}}.

\begin{table}[t]
	\caption{Lines of manually written code of each file for creating a Papyrus UML profile and editor for ArchiMate.}
	\centering
	\setlength{\tabcolsep}{3.5pt} 
	\begin{tabular}{M{2cm}|c|M{1.5cm}|c|M{1.9cm}|}
		\cline{2-5}
		& \multicolumn{3}{c}{\textbf{AMIGO}} & \textbf{Archimate for Papyrus}\\ \hline
		\multicolumn{1}{|M{2cm}|}{\textbf{File}} & \textbf{Handwritten} & \textbf{Handwritten (Polishing)} & \textbf{Total} & \textbf{Total Handwritten}\\ \hline
		\multicolumn{1}{|M{2cm}|}{\textbf{ECore}} & 436 & 0 & 436 & 0 \\ \hline
		\multicolumn{1}{|M{2cm}|}{\textbf{Profile}} & 0 & 0 & 0 & 1867 \\ \hline
		\multicolumn{1}{|M{2cm}|}{\textbf{Palette Configuration}} & 0 & 24 & 24 & 1305 \\ \hline
		\multicolumn{1}{|M{2cm}|}{\textbf{Element Types Configuration}} & 0 & 11 & 11 & 237 \\ \hline
		\multicolumn{1}{|M{2cm}|}{\textbf{Types Configuration}} & 0 & 10 & 10 & 788 \\ \hline
		\multicolumn{1}{|M{2cm}|}{\textbf{Diagram Configuration}} & 0 & 0 & 0 & 58 \\ \hline
		\multicolumn{1}{|M{2cm}|}{\textbf{CSS}} & 0 & 195 & 195 & 553 \\ \hline
		%\multicolumn{1}{|M{2cm}|}{\textbf{plugin.xml}} & 0 & & & 82 \\ \hline
		%\multicolumn{1}{|M{2cm}|}{\textbf{MANIFEST.MF}} & 0 & & & \\ \hline
		\multicolumn{1}{|M{2cm}|}{\textbf{Total}}  & \textbf{436} &\textbf{240} & \textbf{676} & \textbf{4608} \\ \hline
		\cline{1-5}
	\end{tabular}
	\label{tab:evaluation}
\end{table}

As one can see from the numbers, our approach requires about 90\% less 
handwritten code to produce the basic diagrams and about 85\% less code to 
produce the polished editor that matches the original Archimate for Papyrus 
editor. Our approach offers an editor that matches the original Archimate for 
Papyrus tool but also atop that the ETL transformation that allows the 
transformation of the created UML models back to EMF. The generation of OCL 
constraints is also an extra feature that our approach offers. In this 
scenario, however, the generation of constraints could not be assessed as the 
tool that we are comparing with (i.e., Archimate for Papyrus) allows any edge 
to connect any two types of nodes. We leave this evaluation as part of our 
future work.

\subsection{Completeness}
\label{sec:completenessEvaluation}
In addition to the generation of the Archimate profile/editors, we tested the 
proposed approach with nine more Ecore metamodels from different domains. The 
names of the metamodels (including Archimate) and their size (in terms of 
types) are given in Table~\ref{tab:metamodels}. Next to the size, in 
parenthesis, the number of types that should be transformed so they can be 
instantiated as nodes/edges is also provided.

\begin{table}[t]
	\caption{The names and sizes of the ten metamodels against which the approach was evaluated to test completeness}
	\centering
	\setlength{\tabcolsep}{3.5pt} 
	\begin{tabular}{|c|M{2cm}|c|M{2cm}|}
		\cline{1-4}
		\textbf{Name}  & \textbf{\#Types (\#Nodes/\#Edges)} & \textbf{Name}  & \textbf{\#Types (\#Nodes/\#Edges)}\\ \hline
		\textbf{Professor} & 5 (4/5)  & \textbf{Ant Scripts} & 11 (6/4) \\ \hline
		\textbf{Zoo} & 8 (6/4) & \textbf{Cobol} & 13 (12/14) \\ \hline
		\textbf{Usecase} & 9 (4/4) & \textbf{Wordpress} & 20 (19/18)  \\ \hline
		\textbf{Conference} & 9 (7/6) & \textbf{BibTeX} & 21 (16/2) \\ \hline
		\textbf{Bugzilla} & 9 (7/6) & \textbf{Archimate} & 57 (44/11) \\ \hline
		\cline{1-4}
	\end{tabular}
	\label{tab:metamodels}
	
	\vspace*{-3mm}
\end{table}

As illustrated in Table~\ref{tab:metamodels}, the metamodels varied in size, 
from small profiles (having 5 stereotypes) to large profiles (up to 57 
stereotypes). The approach was able to produce the profiles and the editors for 
\textit{all} the metamodels, demonstrating that it can be used to generate the 
desired artifacts for a wide spectrum of domains. The time needed for the 
generation varied from miliseconds up to a few seconds. In the future, we plan 
to assess further the scalability of our approach using larger metamodels.


\subsection{Threats to Validity}
There were a few minor features of the original Archimate for Papyrus tool that our approach could not support. Most of them are related to custom menu entries and wizards. For those to be created developer needs to extend the ``plugin.xml'' file. In addition, the line decoration shapes of stereotypes that extend the aggregation base element (i.e., diamond) can only be applied dynamically by running Java code that will update the property each time the stereotype is applied. Our default and polishing transformations are not able to generate those features automatically; these should be implemented manually. For that reason, we \textit{excluded} these lines of code needed by Archimate for Papyrus to implement these features from the data provided in Table~\ref{tab:evaluation} to have a fair comparison. 

%The implementation of our approach was driven initially by examining the Archimate for Papyrus tool. Although we believe that Archimate for Papyrus offers a complete UML profile solution, there might be other editors that offer other functionality which our approach cannot offer. As a result, the efficiency gain reported might not be that substantial when comparing with other editors. \dk{This can be a showstopper as we're evaluating against the main example we used to develop the tool. I don't think there's a reason to alarm the reviewers.}