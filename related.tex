\section{Related Work} 
\label{sec:related}

%\noindent \textcolor{blue}{A few words about the use of UML profiles}\\\
UML profiles are equipped with suitable extension mechanisms (e.g., 
stereotypes, tagged 
definitions and constraints) that enable the customisation of UML  and 
facilitate the development of DSMLs~\cite{UML2015OMG}. 
Over the past years, several UML profiles have been standardised by the OMG 
(e.g., MARTE~\cite{omg2011marte}, SySML~\cite{friedenthal2014practical}) and 
are now included in most common UML tools (e.g., 
Papyrus~\cite{lanusse2009papyrus}. 
Many researchers, taking advantage of the open-ended boundaries of UML 
profiles, have developed profiles in applications as diverse as Web 
development~\cite{Moreno2007:IETS}, security 
specification~\cite{Bouaziz2012:ICCSE,Rodriguez2010:SERENE}, verification of 
railway systems~\cite{Bernardi2013:JRESS} context modelling (e.g., a 
meeting system) in mobile distributed systems~\cite{Simons2007:JVLC}, and smart 
homecare services~\cite{Walderhaug2009:MODELS}.
Several other researchers have designed UML profiles for the 
specification~\cite{Debnath2006:ICCSA,Mak2004:ICSE} and visualisation of 
design patterns~\cite{Dong2007:TSE}. A list of recently published UML 
profiles is available in~\cite{Pardillo2010:MODELS}. 
Irrespective of the way these UML profiles were developed, either 
following ad-hoc processes or based on guidelines for designing well-structured 
UML profiles~\cite{FuentesFernandez2004:UMLME,Selic2007:ISORC},
they required substantial designer effort. Also, the learning 
curve for new designers interested in exploring whether UML profiles suit their 
business needs is steep.
Our approach, subject to the concerns raised in Section~\ref{sec:evaluation}, 
automates the process of generating such profiles and reduces significantly the 
designer-driven effort for specifying, designing and validating UML Papyrus 
profiles. 

%A manual definition of a UML profile typically is a tedious, time-consuming 
%and error-prone process

%Useful guidelines for designing well-structured UML profiles are proposed 
%in~\cite{FuentesFernandez2004:UMLME,Lagarde2008:FASE}.

%in several domains including designing and developing Web 
%applications~\cite{Moreno2007:IETS}, engineering secure applications
%~\cite{Bouaziz2012:ICCSE}, and modelling and analysing real-time 
%embedded systems~\cite{omg2011marte}.


%\textcolor{blue}{Approaches that automate UML profile generation}
Relevant to our work is research introducing methodologies for the automatic 
generation of UML profiles from an Ecore-based metamodel. 
The work in~\cite{Lagarde2008:FASE} proposes a partially automated approach for 
generating UML profiles using a set of specific design patterns. However, this 
approach requires the manual definition of an initial UML profile skeleton, 
which is typically a tedious and error-prone task~\cite{Wimmer2009:IJWIS}. 
The methodology introduced in~\cite{Giachetti2008:ER,Giachetti2009:CAISE} 
facilitates the derivation of a UML profile using a DSML as input. 
The methodology requires the manual definition of an intermediate metamodel 
that captures the abstract syntax to be integrated into a UML profile, and 
automates the comparison of this intermediate metamodel against the UML 
metamodel to automatically identify a set of required UML extensions, as well 
as the transformation of the intermediate metamodel into a corresponding 
functioning UML profile. 
Despite the potential of these approaches, they usually involve
non-trivial human-driven tasks, e.g., a UML profile 
skeleton~\cite{Lagarde2008:FASE} or an intermediate 
metamodel~\cite{Giachetti2008:ER,Giachetti2009:CAISE}. In contrast, our 
approach builds on top of standard Ecore metamodels (which are usually 
available in MBSE). Furthermore, our approach supports the customisation of  
UML profiles, and the corresponding Papyrus plugin, through the definition of 
optional polishing transformations (cf. 
Section~\ref{sec:transformationPatches}).


%~\cite{Bergmayr2014:MODELS}



Our work also subsumes research that focuses on bridging the gap between 
MOF-based metamodels (e.g., Ecore) and UML profiles.
% and illustrating how derived models can be used interchangeably. 
In~\cite{abouzahra2005practical}, the authors propose a methodology 
that consumes a UML profile and its corresponding Ecore metamodel, and uses
M2M transformation and model weaving to transform UML models to 
Ecore models, and vice versa. The methodology proposed 
in~\cite{Wimmer2009:IJWIS} simplifies the specification of mappings 
between a profile and its corresponding Ecore metamodel using a dedicated 
bridging language. Through an automatic generation process that consumes
these mappings, the technique produces UML profiles and suitable model 
transformations. 
Along the same path, the approach in~\cite{Giachetti2009:ICRCIS} employs an 
integration metamodel to facilitate the interchange of modelling information 
between Ecore-based models and UML models. Compared to this research, our 
approach automatically generates UML 
profiles (like~\cite{Wimmer2009:IJWIS} 
and~\cite{Giachetti2009:ICRCIS}), but requires only a single annotated Ecore 
metamodel and does not need any mediator 
languages~\cite{Wimmer2009:IJWIS} 
or integration metamodels~\cite{Giachetti2009:ICRCIS}. Also, the transformation 
of models from UML profiles to Ecore is only a small part of our generic 
approach~(Section \ref{sec:uml2emf}) that generates not only a fully-fledged 
UML profile but also a distributable custom graphical editor. 
%; we are currently working towards the opposite transformation. 

 


%The issue of defining a bridge between specific technical spaces and MDA has 
%been addressed in several works, such as

%\ \\JUMP [8]  is a tool to reverse engineer Java programs (with annotations)
%into profiled UML class diagrams.
%\ \\ a fully automatic approach to generate profiles from annotation-based 
%Java libraries\cite{Bergmayr2014:MODELS}

%In \cite{langer2011uml}, the authors propose the adaption of the UML Profiles 
%concept into EMF. More specifically, they advocate the use of EMF Profiles as 
%an easier extension mechanism of DSMLs. A metamodel for defining EMF Profiles 
%is proposed along with re-usable EMF Profiles that are developed to tackle 
%specific cases.


%\noindent \textcolor{blue}{Tools that support domain-specific graphical 
%editors}
%\ \\Sirius~\cite{viyovic2014sirius}
%\ \\Eugenia~\cite{kolovos2015eugenia}




