\section{Related Work} 
\label{sec:related}

\subsection{UML Profiles}
Building on the powerful concepts and semantics of UML, and its wide adoption in modelling artefacts of object oriented software and systems, UML profiles enable the development of DSLs by extending (and constraining) elements from the UML metamodel~\cite{FuentesFernandez2004:UMLME}. 
More specifically, UML profiles make use of extension mechanisms (e.g., stereotypes, tagged definitions and constraints) through which engineers can 
specialise generic UML elements and define DSLs that conform to the concepts and nature of specific application domains~\cite{Selic2007:ISORC}. 
Compared to creating a tailor-made DSL by defining its metamodel and developing supporting tools from scratch, the use of UML profiles introduces several benefits including lightweight language extension, dynamic model extension, model-based representation, preservation of metamodel state and employment of already available UML-based tools~\cite{langer2011uml}. 
Driven by these benefits, several UML profiles have been standardised by the OMG including MARTE~\cite{omg2011marte} and SySML~\cite{friedenthal2014practical}) which are now included in most widely used UML tools (e.g., Papyrus~\cite{lanusse2009papyrus}). 

The flexibility and open-ended boundaries of UML profiles facilitated the development of profiles in applications as diverse as performance analysis~\cite{xu2003performance} and Quality-of-Service investigation~\cite{cortellessa2004towards} in component-based systems, as well as context modelling in mobile distributed systems~\cite{Simons2007:JVLC},  Web applications~\cite{Moreno2007:IETS} and smart homecare services~\cite{Walderhaug2009:MODELS}. In safety-critical application domains
such as railway, avionics and network infrastructures, developed UML profiles support the specification and examination of security patterns~\cite{Bouaziz2012:ICCSE,Rodriguez2010:SERENE}, analysis of intrusion detection scenarios~\cite{hussein2006umlintr}, and modelling and verification of safety-critical software~\cite{Bernardi2013:JRESS,zoughbi2007uml}, 

Other researchers have designed UML profiles for the 
specification~\cite{Debnath2006:ICCSA,Mak2004:ICSE} and visualisation of 
design patterns~\cite{Dong2007:TSE}. 
Also,~\cite{tatibouet2014formalizing} proposes a methodology for formalising the semantics of UML profiles based on fUML~\cite{fuml}, a subset of UML
limited to composite structures, classes and activities with a precise execution semantics. 
%A list of recently published UML profiles is available in~\cite{Pardillo2010:MODELS}. 
For an analysis of qualitative characteristics of several UML profiles and a discussion of adopted practices for UML profiling definition, see~\cite{Pardillo2010:MODELS}.
Likewise, interested readers can find a comprehensive review on execution of UML and UML profiles in~\cite{ciccozzi2018execution}.

Irrespective of the way these UML profiles were developed, either 
following ad-hoc processes or based on guidelines for designing well-structured 
UML profiles~\cite{FuentesFernandez2004:UMLME,Selic2007:ISORC},
they required substantial designer effort. Also, the learning 
curve for new designers interested in exploring whether UML profiles suit their 
business needs is steep~\cite{Giachetti2009:CAISE}.
In contrast, Jorvik automates the process of generating UML profiles using a single annotated Ecore metamodel and reduces significantly the developer's effort for specifying, designing and validating UML Papyrus profiles (cf. Section~\ref{sec:evaluation}).

%A manual definition of a UML profile typically is a tedious, time-consuming 
%and error-prone process
\subsection{Automatic Generation of UML Profiles}
Relevant to Jorvik is research introducing methodologies for the automatic 
generation of UML profiles from an Ecore-based metamodel~\cite{Kraas17}. 
The work in~\cite{Lagarde2008:FASE} proposes a partially automated approach for 
generating UML profiles using a set of specific design patterns. However, this 
approach requires the manual definition of an initial UML profile skeleton, 
which is typically a tedious and error-prone task~\cite{Wimmer2009:IJWIS}. 
The methodology introduced in~\cite{Giachetti2008:ER,Giachetti2009:CAISE} 
facilitates the derivation of a UML profile using a simpler DSL as input.
The methodology requires the manual definition of an intermediate metamodel 
that captures the abstract syntax to be integrated into a UML profile. 
The intermediate metamodel is then compared against the UML 
metamodel to identify a set of required UML extensions, as well 
as the transformation of the intermediate metamodel into a corresponding 
functioning UML profile. 
Similarly,~\cite{Kraas17} introduces an approach for the automatic derivation of a UML profile and a corresponding set of OCL expressions
for stereotype attributes using annotated MOF-based metamodels.
Another relevant research work is JUMP~\cite{Bergmayr2014:MODELS} that supports the automatic generation of profiles from annotated Java libraries~\cite{Bergmayr2014:MODELS}.
Despite the potential of these approaches, they usually involve
non-trivial human-driven tasks, e.g., a UML profile 
skeleton~\cite{Lagarde2008:FASE} or an intermediate 
metamodel~\cite{Giachetti2008:ER,Giachetti2009:CAISE}, or have limited capabilities (e.g., support of UML profile derivation with generation of OCL constraints~\cite{Kraas17}). In contrast, Jorvik builds on top of standard Ecore metamodels that form the building blocks of MDE~\cite{omg2014meta}. 
Furthermore, Jorvik facilitates the development of a fully-fledged UML profile and a distributable Papyrus graphical editor including the generation of OCL constraints and the definition of optional polishing transformations (cf. 
Section~\ref{sec:transformationPatches}).
%supports the customisation of UML profiles including the generation of OCL constraints, as well as the generation of the corresponding Papyrus plugin, through the definition of optional polishing transformations (cf. Section~\ref{sec:transformationPatches}).


\subsection{From Ecore to UML profiles and back}
Jorvik also subsumes research that focuses on bridging the gap between 
MOF-based metamodels (e.g., Ecore) and UML profiles.
% and illustrating how derived models can be used interchangeably. 
In~\cite{abouzahra2005practical}, the authors propose a methodology 
that consumes a UML profile and its corresponding Ecore metamodel, and uses
M2M transformation and model weaving to transform UML models to 
Ecore models, and vice versa. The methodology proposed 
in~\cite{Wimmer2009:IJWIS} simplifies the specification of mappings 
between a profile and its corresponding Ecore metamodel using a dedicated 
bridging language. Through an automatic generation process that consumes
these mappings, the technique produces UML profiles and suitable model 
transformations. 
Along the same path, the approach in~\cite{Giachetti2009:ICRCIS} employs an 
integration metamodel to facilitate the interchange of modelling information 
between Ecore-based models and UML models. Compared to this research, Jorvik automatically generates UML 
profiles (like~\cite{Wimmer2009:IJWIS} 
and~\cite{Giachetti2009:ICRCIS}), but requires only a single annotated Ecore 
metamodel and does not need any mediator 
languages~\cite{Wimmer2009:IJWIS} 
or integration metamodels~\cite{Giachetti2009:ICRCIS}. Also, the transformation 
of models from UML profiles to Ecore is only a small part of our generic 
approach~(Section \ref{sec:uml2emf}) that generates not only a fully-fledged 
UML profile but also a distributable custom graphical editor as an Eclipse plugin. 

 


%The issue of defining a bridge between specific technical spaces and MDA has 
%been addressed in several works, such as


%In \cite{langer2011uml}, the authors propose the adaption of the UML Profiles 
%concept into EMF. More specifically, they advocate the use of EMF Profiles as 
%an easier extension mechanism of DSMLs. A metamodel for defining EMF Profiles 
%is proposed along with re-usable EMF Profiles that are developed to tackle 
%specific cases.
