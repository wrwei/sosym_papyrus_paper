\section{Conclusions and Future Work}
\label{sec:future}
In this paper we presented an approach towards automatic generation of UML profiles and supporting distributable Papyrus editors from an annotated Ecore metamodel. Our approach automatically generates the appropriate UML profile and all the needed artefacts for a fully functional Papyrus editor for the profile. In addition, it allows users to override/complement the built-in transformations to further polish the generated editor.

In the current version, although users are able to create compartments using an already existing compartment relationships in UML (e.g., Package-Class, Class-Property, etc.) the visual result is not appealing. More specifically, the compartment where containing elements are placed is distinct and lies above the compartment that hosts the shape. As a result the contained elements are drawn above the custom shape and not inside it. In the future, we plan to better support compartments by overriding the way in which these are displayed in Papyrus. In addition, in the current version we only support the automatic generation of OCL constraints for connectors for the \textit{Association} base element. In the future more connectors, as well as opposite references will be supported. 

%In addition, for the stereotypes generated by EReferences, the EReference itself remains a property of the stereotype. This was done for pragmatic reasons: in case users wish to perform model management activities on the models they create using the generated editor, it is easier to retrieve the targets of the references using the EReference rather than navigating through the stereotyped edge using Epsilon. However, the stereotype that represents the EReference and the EReference itself are not synchronised: if the edge is drawn on the canvas to connect two elements the reference collection of the source element is not automatically populated with the target node. Those should be synchronised manually \dk{This is too immature to even try to explain as a limitation. I'd recommend removing this paragraph}. This could be automated by generating and attaching listeners that call methods to populate the references. Another approach would be to simplify the way target elements can be retrieved by Epsilon scripts via the stereotyped edges. 

Finally, the annotations provided in the Ecore file, might contain errors (e.g., users might point to a base class that does not exist). Validation scripts should be run against the Ecore file to check for such errors and produce meaningful error messages to the user in the future.