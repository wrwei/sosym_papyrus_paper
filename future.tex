\section{Conclusions and Future Work}
\label{sec:future}
In this paper we presented an approach for automatic generation of UML profiles and their supporting distributable Papyrus editors from annotated Ecore metamodels. 
Our approach automatically generates the appropriate UML profile and all the needed artefacts for a fully functional Papyrus editor for the profile. 
In addition, it allows users to override/complement the built-in transformations to further polish the generated editor.

We evaluated the efficiency of Jorvik in terms of the amount of effort required through replicating what Archimate for Papyrus does. 
We evaluated the completeness of Jorvik in terms of its wide applicability through generating a number of Papyrus editors for a variety of EMF metamodels.
We evaluated the boost in productivity of Jorvik in terms of the amount of time required through a user experiment.
We conclude that Jorvik is a complete solution, which improves the efficiency and boosts the productivity for developers in creating UML profiles and their supporting editors in Papyrus.

%In the current version, although users are able to create compartments using an already existing compartment relationships in UML (e.g., Package-Class, Class-Property, etc.) the visual result is not appealing. 
%More specifically, the compartment where containing elements are placed is distinct and lies above the compartment that hosts the shape. 
%As a result the contained elements are drawn above the custom shape and not inside it. 
%In the future, we plan to better support compartments by overriding the way in which these are displayed in Papyrus. \thanos{We did in this work. IMPORTANT: did Horacio write in the paper about this? We need to write because we list this as a contibutution of this paper.}
In the current version we only support the automatic generation of OCL constraints for connectors for the \textit{Association} base element.
In the future work, we will try to support other connector types, such as Dependency and Composition. 
Currently, our approach is a one way transformation from annotated Ecore metamodels to UML profiles and their editors. 
In the future work, we plan to support the generation of editors based on a UML profile. 
In addition, since there was a major change in infrastructure of Papyrus version 3.0.0, we plan to build a migration tool, which uses the generation of editors based on UML profiles and support the migration of editors built for pre-Papyrus 3.0.0 to Papyrus 3.0+.
\\
%In addition, for the stereotypes generated by EReferences, the EReference itself remains a property of the stereotype. This was done for pragmatic reasons: in case users wish to perform model management activities on the models they create using the generated editor, it is easier to retrieve the targets of the references using the EReference rather than navigating through the stereotyped edge using Epsilon. However, the stereotype that represents the EReference and the EReference itself are not synchronised: if the edge is drawn on the canvas to connect two elements the reference collection of the source element is not automatically populated with the target node. Those should be synchronised manually \dk{This is too immature to even try to explain as a limitation. I'd recommend removing this paragraph}. This could be automated by generating and attaching listeners that call methods to populate the references. Another approach would be to simplify the way target elements can be retrieved by Epsilon scripts via the stereotyped edges. 

