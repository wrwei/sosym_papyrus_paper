\begin{abstract}
UML profiles offer a way for developers to build domain-specific modelling languages by re-using and extending UML concepts.
Eclipse Papyrus is a powerful open-source UML modelling tool which supports UML profiling.
However, with power comes complexity -- implementing non-trivial UML profiles and editors for them typically requires the developers to hand craft and maintain a number of interconnected models through a loosely guided, labour-intensive and error-prone process.
We demonstrate how metamodel annotations and model transformation techniques can help to manage the complexity of Papyrus in the creation of UML profiles and their dedicated editors. 
We present Jorvik, an open-source tool that implements the proposed approach. 
We illustrate its functionality with examples, and we evaluate our approach by comparing it against manual UML profile specification and implementation using a non-trivial enterprise modelling language (Archimate) as a case study. We also perform a user study in which developers are asked to produce identical editors using both Papyrus and Jorvik demonstrating the substantial productivity and maintainability benefits that Jorvik delivers.
\end{abstract}