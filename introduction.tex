\section{Introduction}
\label{sec:introduction}

The Unified Modelling Language (UML)~\cite{UML2015OMG} is the \emph{de facto} standard for object-oriented software and systems modelling. 
It offers a broad range of abstractions that can be used to express different views of a system, including Class, Use Case, State, Collaboration and Sequence diagrams. 
Since version 2.0, UML offers a extension and customisation mechanism named \emph{UML Profiling}~\cite{FuentesFernandez2004:UMLME}.
UML profiling enables the users to derive Domain-Specific Languages (DSL) from UML's set of general language concepts.
An important advantage of this approach to DSL design is that it allows the reuse of existing UML tools and widely available UML expertise.
The basic premise of profiles is that all domain specific concepts are derived as extensions or refinements of existing UML concepts, called UML \textit{meta-classe}s. 
These extensions are called \textit{Stereotype}s. 
A \textit{Stereotype} definition must be consistent with the abstract syntax and semantics of standard UML \textit{meta-classe}s it extends. 
Consequently, a profile-based model can be created and manipulated by any tool that supports standard UML. 
Moreover, because the concepts underlying a profile specialisations of existing UML concepts, users with UML knowledge can adapt to the approach more easily.

%Using UML Profiles users can add new concepts to the modelling language in the form of \textit{Stereotypes}. 
%Each Stereotype extends a core UML concept (e.g., Class, Actor, etc.) and includes extra information that 
%supports the modelling needs of users of the target domain that is missing from the base concept.

%With profiles, UML offers a way for users to build Domain-Specific Modelling Languages (DSML) by re-using and extending UML concepts. 
Although domain-specific modelling languages and tools that support them, like Sirius~\cite{viyovic2014sirius} or Eugenia~\cite{kolovos2015eugenia}, are becoming more popular, UML is still widely used in model-based software engineering (MBSE)~\cite{erickson2007theoretical}. 
As a result, alternative ways to define domain-specific languages using dedicated metamodels and textual/graphical editors are available to the users~\cite{Bergmayr2014:MODELS,Pardillo2010:MODELS}. 
%Those approaches have their benefits and shortcomings which are, however, beyond the scope of this paper.   

Papyrus \cite{lanusse2009papyrus} is a leading open-source UML modelling tool developed under the Eclipse Foundation and driven by the PolarSys Initiative and the Papyrus Industry Consortium, which are spearheaded by large high-technology companies such as Airbus, Thales, Saab and Ericsson. \will{Is this still the case?} \thanos{I was told that this is not going to be the case, but I don't know if they have made any progress. You might need to check online or in the forum.}
After more than a decade in development, Papyrus is close to developing a critical mass for wider adoption in industry as means of (1) escaping proprietary UML tooling lock-in, (2) leveraging the MBSE-related developments in the Eclipse modelling ecosystem enabling automated management of UML models (e.g. model validation and model-to-model (M2M) and model-to-text (M2T) transformation languages), and (3) enabling multi-paradigm modelling using a combination of UML and EMF-based DSLs. 
Widely used UML profiles, like SysML~\cite{friedenthal2014practical} and MARTE~\cite{omg2011marte} offer implementations for Papyrus. 
However, the ability of Papyrus to support non-trivial UML profiles and the effort and learning curve related to developing such profiles are recurring concerns. 
As reported in~\cite{Wimmer2009:IJWIS}, the manual definition of new UML profiles is typically a tedious, time-consuming and error-prone process.

In this paper, we simplify and automate the process of developing distributable Papyrus editors for UML profiles. 
We propose an approach supported by a prototype Eclipse plugin, where annotated Ecore metamodels are used to capture the abstract and graphical syntax of profiles at a high-level of abstraction, and are then automatically transformed to the Papyrus-specific artefacts for distribution and deployment. 
In particular, our approach, called Jorvik, produces a fully-fledged UML profile and a distributable Papyrus graphical editor as an Eclipse plugin. 

We evaluate the effectiveness of our approach for the automatic generation of a non-trivial enterprise modelling language (Archimate) and its corresponding Papyrus editor against an equivalent developer-driven UML profile and Papyrus editor. 
Furthermore, we apply our approach on several other DSMLs of varying size and complexity~\cite{williams2013metamodels}, demonstrating its generality and wide applicability. We also evaluate our approach in a user study where developers are asked to created distributable Papyrus editors for two UML profiles using Papyrus' dedicated infrastructure and Jorvik, in order to compare the efficiency of our proposed approach.

This paper is an extended version of the work presented in~\cite{zolotas2018towards}. Compared to~\cite{zolotas2018towards}, in this paper we also:
\begin{enumerate}
	\item{Performed a major refactoring of the tool to support the adapt to the new Papyrus 3.0 structure.}
	\item{Executed user experiments to compare our proposed approach with the default Papyrus approach.}
	\item{Developed a validation script that checks the annotated ECore metamodels given as input to our approach.}
	\item{Added more styling properties for diagrams created with our approach.}
	\item{Explored the feasibility of having nested elements in Papyrus graphical editors.} \thanos{@Horacio, not sure how to write this properly. Could you please fix this?}
\end{enumerate}
The rest of the paper is organised as follows. 
Section~\ref{sec:background} motivates the need of the proposed approach. 
Section~\ref{sec:approach} describes the proposed approach while in Section~\ref{sec:implementation} the implementation details are given. 
In Section~\ref{sec:evaluation} we present the evaluation carried out, while in Section~\ref{sec:related} related work is discussed. 
Finally, Section~\ref{sec:future} concludes the paper and highlights plans for future work.