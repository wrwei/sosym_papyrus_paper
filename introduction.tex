\section{Introduction}
\label{sec:introduction}

The Unified Modelling Language (UML)~\cite{UML2015OMG} is the \emph{de facto} standard for object-oriented software and systems modelling. 
It offers a broad range of abstractions that can be used to express different views of a system, including Class, Use Case, State, Collaboration and Sequence diagrams. 
Since version 2.0, UML offers an extension and customisation mechanism named \emph{UML Profiling}~\cite{FuentesFernandez2004:UMLME}.
UML profiling enables the users to derive Domain-Specific Languages (DSL) from UML's set of general language concepts.
An important advantage of this approach to DSL design is that it allows the reuse of existing UML tools and it supports widely available UML expertise.
The basic premise of UML profiling is that all domain specific concepts are derived as extensions or refinements of existing UML concepts (called UML \textit{meta-element}s). 
These extensions are called \textit{Stereotype}s. 
A \textit{Stereotype} definition must be consistent with the abstract syntax and semantics of standard UML \textit{meta-element}s it extends. 
Consequently, a profile-based model can be created and manipulated by any tool that supports standard UML. 
Moreover, the concepts underlying profile specialisations of existing UML concepts enables users with UML knowledge to adapt to the approach more easily.

%With profiles, UML offers a way for users to build Domain-Specific Modelling Languages (DSML) by re-using and extending UML concepts. 
Although Domain-Specific Modelling Languages and tools that support them, like Sirius~\cite{viyovic2014sirius} or Eugenia~\cite{kolovos2015eugenia}, are becoming more popular, UML is still widely used in Model-Based Software Engineering (MBSE)~\cite{erickson2007theoretical}. 
As a result, alternative ways to define Domain-Specific Languages using dedicated metamodels and textual/graphical editors are available to the users~\cite{Bergmayr2014:MODELS,Pardillo2010:MODELS}. 
%Those approaches have their benefits and shortcomings which are, however, beyond the scope of this paper.   

Papyrus \cite{lanusse2009papyrus} is a leading open-source UML modelling tool developed under the Eclipse Foundation and driven by the PolarSys Initiative and the Papyrus Industry Consortium, which are spearheaded by large high-technology companies such as Airbus, Thales, Saab and Ericsson. 
After more than a decade in development, Papyrus is close to developing a critical mass for wider adoption in industry as means of 1) escaping proprietary UML tooling lock-in, 2) leveraging the MBSE-related developments in the Eclipse modelling ecosystem enabling automated management of UML models (e.g. model validation and model-to-model (M2M) and model-to-text (M2T) transformation languages), and 3) enabling multi-paradigm modelling using a combination of UML and EMF-based DSLs. 
OMG-compliant UML profiles, like SysML~\cite{friedenthal2014practical} and MARTE~\cite{omg2011marte} offer implementations for Papyrus. 
As highlighted in the recent systematic survey on execution of UML models and UML profiles~\cite{ciccozzi2018execution}, Papyrus is the most widely used tool for developing UML profiles.
However, the ability of Papyrus to support non-trivial UML profiles, altogether with the effort and learning curve related to developing such profiles are recurring concerns. 
As reported in~\cite{Wimmer2009:IJWIS}, the manual definition of new UML profiles is typically a tedious, time-consuming and error-prone process.

Papyrus also supports the creation of profile-specific graphical editors which enables the users to define their own creation palettes, custom styles and related artefacts based on their own UML profiles. 
However, the process of creating profile-specific graphical editors is typically difficult because it requires a high level of modelling expertise and it can also be a repetitive and error-prone process.

In this paper, we simplify and automate the process of developing distributable Papyrus UML profile specific editors. 
We present \textit{Jorvik}, an approach supported by an Eclipse Plug-in, which enables the use of annotated Ecore metamodels to capture the abstract and graphical syntax of UML profiles at a high-level of abstraction, and are then automatically transformed to UML profiles, and artefacts that contribute to distributable Papyrus graphical editors based on the UML profiles. 


We evaluate the efficiency of Jorvik for the automatic generation of a non-trivial enterprise modelling language (Archimate) and its corresponding Papyrus editor against an equivalent developer-driven UML profile and its Papyrus editor. 
We then apply our approach on several other DSMLs of varying size and complexity~\cite{williams2013metamodels}, to demonstrate its generality and wide applicability. 
Furthermore, we evaluate the productivity boost of our approach in a user study where developers are asked to create two UML profiles and their Papyrus graphical editors manually and with Jorvik.
We report the findings in our study and we report the efficiency of Jorvik based on the results. 

This paper significantly extends the prototype approach for automated generation of UML profile graphical editors for Papyrus from our conference paper~\cite{zolotas2018towards} in the following ways:
\begin{enumerate}
	\item \textit{We perform a refactoring of Jorvik to adapt to the new underlying structure of Papyrus 3.0 and above.} --
	Since Papyrus 3.0, its underlying metamodels have changed, and the process for creating a Papyrus editor has changed significantly. 
	We therefore re-implemented a majority of our work to adapt to the changes. 
	\item \textit{We enhance considerably the experimental evaluation by conducting user experiments to compare Jorvik with the default Papyrus approach.} --
	Significant time has been spent on studying the efficiency of our approach, compared to the manual approach.
	\item \textit{We develop a validation script that checks the annotated Ecore metamodels given as input to our approach.} -- The validation script provides useful feedback to the users to assist them in creating correct annotations.
	\item \textit{We support more styling properties for diagrams created with our approach.} -- We add support for font styling and we add support for line styling.
%	\item \textit{We explore the feasibility of having nested elements in Papyrus graphical editors when using custom shapes.} \thanos{@Horacio, not sure how to write this properly. Could you please fix this?}
\end{enumerate}
The rest of the paper is organised as follows. 
In Section~\ref{sec:background} we motivate the need of the proposed approach. 
In Section~\ref{sec:approach} we describe the proposed approach while 
in Section~\ref{sec:implementation} we discuss in detail the artefacts needed to create a working Papyrus editor, and the implementation details of automatically generating these artefacts. 
In Section~\ref{sec:evaluation} we present the evaluations we conducted.
In Section~\ref{sec:related}, we discuss related work and finally, in Section~\ref{sec:future} we conclude the paper and highlight the plans for future work.